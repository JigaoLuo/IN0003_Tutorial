\documentclass[
  %solution,
  english
	%german
]{tumteaching}

\usepackage{listings}
\usepackage{paralist}

\usepackage{mathpartir}
\usepackage{stmaryrd}
\usepackage{braket}
\usepackage[inline]{enumitem}
\usepackage{amsmath,amsthm}
\usepackage{multirow}
\usepackage{url}
\usepackage[dvipsnames]{xcolor}
\usepackage{centernot}
\usepackage{multicol}
\usepackage{adjustbox}
\usepackage{ltablex}

\usepackage{info2}

\usetikzlibrary{decorations.pathmorphing}
\usetikzlibrary{shapes.callouts}

\tikzset{snake it/.style={decorate, decoration=snake}}

\ExplSyntaxOn%
\tl_gset:Nn \g_tumteaching_date_tl {WS~2018/19}
\tl_gset:Nn \g_tumteaching_exsheet_nr_tl {3}
\tl_gset:Nn \g_tumteaching_exsheet_deadline_tl {11.11.2018}
\ExplSyntaxOff%

\newcommand{\objective}[1]{\item \small{#1}}
%\newcommand{\mydottedline}[0]{...}
\newcommand{\mydotline}{\makebox[2.0in]{\dotfill}}
\newcolumntype{C}{>{\centering\arraybackslash}X}


\begin{document}

\verticalline
\begin{disclaimer}{General Information}
	Detailed information about the lecture, tutorials and homework assignments can be found on the lecture website\footnote{\url{https://www.in.tum.de/i02/lehre/wintersemester-1819/vorlesungen/functional-programming-and-verification/}}. Solutions have to be submitted to Moodle\footnote{\url{https://www.moodle.tum.de/course/view.php?id=44932}}. Make sure your uploaded documents are readable. Blurred images will be rejected. Use Piazza\footnote{\url{https://piazza.com/tum.de/fall2018/in0003/home}} to ask questions and discuss with your fellow students.
\end{disclaimer}

\verticalline

\begin{disclaimer}{Loop Invariants}
	In this exercise sheet, you will discuss and practice different strategies to find suitable loop invariants. For additional insight, tips and tricks on how to find a good invariant, we recommend watching the recording of last year's exercise on this particular topic\footnote{\url{http://ttt.in.tum.de/recordings/Info2_2017_11_24-1/Info2_2017_11_24-1.mp4}}.
\end{disclaimer}

\verticalline

\begin{assignment}[L]{Individual Loops}
	Inspect the following loops and discuss the preconditions that have to hold, such that the assertion $i = n$ is satisfied. In particular, discuss the results for positive and negative inputs, respectively.
	
	\begin{tabularx}{\textwidth}{XXX}
		1.~~
		
		\begin{tikzpicture}[scale=0.8]
\node[coordinate](start){Start};
\node[below=of start.south, rectangle, draw, yshift=3mm](assignment_0){$i = 0$};
\path[->](start) edge [] node [right]{} (assignment_0);
\node[below=of assignment_0.south, circle, draw, yshift=3mm](loop_1_join){};
\path[->](assignment_0) edge [] node [right]{} (loop_1_join);
\node[below=of loop_1_join.south, diamond, draw, yshift=3mm, scale=0.85](loop_1_fork){$i == n$};
\path[->](loop_1_join) edge [] node [right]{} (loop_1_fork);
\draw let \p1=(loop_1_fork.west) in let \p2=(loop_1_fork.west) in node[coordinate](loop_1_fork_angle_l) at (\x1+-10mm, \y2+0mm) {};
\path[-](loop_1_fork) edge [] node [above]{$yes$} (loop_1_fork_angle_l);
\draw let \p1=(loop_1_fork.east) in let \p2=(loop_1_fork.east) in node[coordinate](loop_1_fork_angle_r) at (\x1+10mm, \y2+0mm) {};
\path[-](loop_1_fork) edge [] node [above]{$no$} (loop_1_fork_angle_r);
\node[above=of loop_1_fork_angle_r.north, rectangle, draw, yshift=-3mm](assignment_2){$i = i+2$};
\path[->](loop_1_fork_angle_r) edge [] node [right]{} (assignment_2);
\draw let \p1=(assignment_2.north) in let \p2=(loop_1_join) in node[coordinate](loop_1_back_edge_angle1) at (\x1+0mm, \y2+0mm) {};
\path[-](assignment_2) edge [] node [right]{} (loop_1_back_edge_angle1);
\path[->](loop_1_back_edge_angle1) edge [] node [right]{} (loop_1_join);
\node[below=of loop_1_fork_angle_l.south, coordinate,yshift=3mm](stop){Stop};
\path[->](loop_1_fork_angle_l) edge [] node [left]{$i = n$} (stop);		
		\end{tikzpicture} &
		2.~~
		
		\begin{tikzpicture}[scale=0.8]
\node[coordinate](start){Start};
\node[below=of start.south, rectangle, draw, yshift=3mm](assignment_0){$i = 0$};
\path[->](start) edge [] node [right]{} (assignment_0);
\node[below=of assignment_0.south, circle, draw, yshift=3mm](loop_1_join){};
\path[->](assignment_0) edge [] node [right]{} (loop_1_join);
\node[below=of loop_1_join.south, diamond, draw, yshift=3mm,scale=0.85](loop_1_fork){$i >= n$};
\path[->](loop_1_join) edge [] node [right]{} (loop_1_fork);
\draw let \p1=(loop_1_fork.west) in let \p2=(loop_1_fork.west) in node[coordinate](loop_1_fork_angle_l) at (\x1+-10mm, \y2+0mm) {};
\path[-](loop_1_fork) edge [] node [above]{$yes$} (loop_1_fork_angle_l);
\draw let \p1=(loop_1_fork.east) in let \p2=(loop_1_fork.east) in node[coordinate](loop_1_fork_angle_r) at (\x1+10mm, \y2+0mm) {};
\path[-](loop_1_fork) edge [] node [above]{$no$} (loop_1_fork_angle_r);
\node[above=of loop_1_fork_angle_r.north, rectangle, draw, yshift=-3mm](assignment_2){$i = i+1$};
\path[->](loop_1_fork_angle_r) edge [] node [right]{} (assignment_2);
\draw let \p1=(assignment_2.north) in let \p2=(loop_1_join) in node[coordinate](loop_1_back_edge_angle1) at (\x1+0mm, \y2+0mm) {};
\path[-](assignment_2) edge [] node [right]{} (loop_1_back_edge_angle1);
\path[->](loop_1_back_edge_angle1) edge [] node [right]{} (loop_1_join);
\node[below=of loop_1_fork_angle_l.south, coordinate,yshift=3mm](stop){Stop};
\path[->](loop_1_fork_angle_l) edge [] node [left]{$i = n$} (stop);		
		\end{tikzpicture}&
		3.~~
		
		\begin{tikzpicture}[scale=0.8]
\node[coordinate](start){Start};
\node[below=of start.south, rectangle, draw, yshift=3mm](assignment_0){$i = 0$};
\path[->](start) edge [] node [right]{} (assignment_0);
\node[below=of assignment_0.south, circle, draw, yshift=3mm](loop_1_join){};
\path[->](assignment_0) edge [] node [right]{} (loop_1_join);
\node[below=of loop_1_join.south, diamond, draw, yshift=3mm,scale=0.85](loop_1_fork){$i >= n$};
\path[->](loop_1_join) edge [] node [right]{} (loop_1_fork);
\draw let \p1=(loop_1_fork.west) in let \p2=(loop_1_fork.west) in node[coordinate](loop_1_fork_angle_l) at (\x1+-10mm, \y2+0mm) {};
\path[-](loop_1_fork) edge [] node [above]{$yes$} (loop_1_fork_angle_l);
\draw let \p1=(loop_1_fork.east) in let \p2=(loop_1_fork.east) in node[coordinate](loop_1_fork_angle_r) at (\x1+10mm, \y2+0mm) {};
\path[-](loop_1_fork) edge [] node [above]{$no$} (loop_1_fork_angle_r);
\node[above=of loop_1_fork_angle_r.north, rectangle, draw, yshift=-3mm](assignment_2){$i = i+2$};
\path[->](loop_1_fork_angle_r) edge [] node [right]{} (assignment_2);
\draw let \p1=(assignment_2.north) in let \p2=(loop_1_join) in node[coordinate](loop_1_back_edge_angle1) at (\x1+0mm, \y2+0mm) {};
\path[-](assignment_2) edge [] node [right]{} (loop_1_back_edge_angle1);
\path[->](loop_1_back_edge_angle1) edge [] node [right]{} (loop_1_join);
\node[below=of loop_1_fork_angle_l.south, coordinate,yshift=3mm](stop){Stop};
\path[->](loop_1_fork_angle_l) edge [] node [left]{$i = n$} (stop);		
		\end{tikzpicture}
	\end{tabularx} 
	
	%@@@ BEGIN_SOLUTION
	%@@@ END_SOLUTION
\end{assignment}


\clearpage
\begin{assignment}[L]{Y?}
Consider these control flow graph fragments (assume $x$ and $y$ to be $0$ initially):
	\begin{tabularx}{\textwidth}{XX}
		1.~~
		
		\begin{tikzpicture}[scale=0.7]
\node[coordinate](start){Start};
\node[below=of start.south, rectangle, draw, yshift=3mm](assignment_0){$i = 0$};
\path[->](start) edge [] node [right]{} (assignment_0);
\node[below=of assignment_0.south, circle, draw, yshift=6mm](loop_1_join){};
\path[->](assignment_0) edge [] node [right]{} (loop_1_join);
\node[below=of loop_1_join.south, diamond, draw, yshift=6mm](loop_1_fork){i == n};
\path[->](loop_1_join) edge [] node [right]{} (loop_1_fork);
\draw let \p1=(loop_1_fork.west) in let \p2=(loop_1_fork.west) in node[coordinate](loop_1_fork_angle_l) at (\x1-10mm, \y2+0mm) {};
\path[-](loop_1_fork) edge [] node [above]{$yes$} (loop_1_fork_angle_l);
\draw let \p1=(loop_1_fork.east) in let \p2=(loop_1_fork.east) in node[coordinate](loop_1_fork_angle_r) at (\x1+10mm, \y2+0mm) {};
\path[-](loop_1_fork) edge [] node [above]{$no$} (loop_1_fork_angle_r);
\node[below=of loop_1_fork_angle_r.south, rectangle, draw, yshift=3mm](assignment_2){$i = i+1$};
\path[->](loop_1_fork_angle_r) edge [] node [right]{} (assignment_2);
\node[below=of assignment_2.south, rectangle, draw, yshift=6mm](assignment_3){$y = 5 * i$};
\path[->](assignment_2) edge [] node [right]{} (assignment_3);
\node[below=of assignment_3.south, rectangle, draw, yshift=6mm](assignment_4){$x = x+y$};
\path[->](assignment_3) edge [] node [right]{} (assignment_4);
\draw let \p1=(assignment_4.south) in let \p2=(assignment_4.south) in node[coordinate](loop_1_back_edge_angle1) at (\x1+0mm, \y2+-4mm) {};
\path[-](assignment_4) edge [] node [right]{} (loop_1_back_edge_angle1);
\draw let \p1=(loop_1_back_edge_angle1) in let \p2=(loop_1_back_edge_angle1) in node[coordinate](loop_1_back_edge_angle2) at (\x1+20mm, \y2+0mm) {};
\path[-](loop_1_back_edge_angle1) edge [] node [right]{} (loop_1_back_edge_angle2);
\draw let \p1=(loop_1_back_edge_angle2) in let \p2=(loop_1_join) in node[coordinate](loop_1_back_edge_angle3) at (\x1+0mm, \y2+0mm) {};
\path[-](loop_1_back_edge_angle2) edge [] node [right]{} (loop_1_back_edge_angle3);
\path[->](loop_1_back_edge_angle3) edge [] node [right]{} (loop_1_join);
\node[below=of loop_1_fork_angle_l.south, coordinate, yshift=3mm](stop){Stop};
\path[->](loop_1_fork_angle_l) edge [] node [left]{$x = {\displaystyle \sum_{k = 0}^{n} 5k}$} (stop);
	\end{tikzpicture}&
		2.~~

		\begin{tikzpicture}[scale=0.7]
\node[coordinate](start){Start};
\node[below=of start.south, rectangle, draw, yshift=3mm](assignment_0){$i = 0$};
\path[->](start) edge [] node [right]{} (assignment_0);
\node[below=of assignment_0.south, circle, draw, yshift=6mm](loop_1_join){};
\path[->](assignment_0) edge [] node [right]{} (loop_1_join);
\node[below=of loop_1_join.south, diamond, draw, yshift=6mm](loop_1_fork){i == n};
\path[->](loop_1_join) edge [] node [right]{} (loop_1_fork);
\draw let \p1=(loop_1_fork.west) in let \p2=(loop_1_fork.west) in node[coordinate](loop_1_fork_angle_l) at (\x1+-10mm, \y2+0mm) {};
\path[-](loop_1_fork) edge [] node [above]{$yes$} (loop_1_fork_angle_l);
\draw let \p1=(loop_1_fork.east) in let \p2=(loop_1_fork.east) in node[coordinate](loop_1_fork_angle_r) at (\x1+10mm, \y2+0mm) {};
\path[-](loop_1_fork) edge [] node [above]{$no$} (loop_1_fork_angle_r);
\node[below=of loop_1_fork_angle_r.south, rectangle, draw, yshift=3mm](assignment_2){$i = i+1$};
\path[->](loop_1_fork_angle_r) edge [] node [right]{} (assignment_2);
\node[below=of assignment_2.south, rectangle, draw, yshift=6mm](assignment_3){$y = y+5$};
\path[->](assignment_2) edge [] node [right]{} (assignment_3);
\node[below=of assignment_3.south, rectangle, draw, yshift=6mm](assignment_4){$x = x+y$};
\path[->](assignment_3) edge [] node [right]{} (assignment_4);
\draw let \p1=(assignment_4.south) in let \p2=(assignment_4.south) in node[coordinate](loop_1_back_edge_angle1) at (\x1+0mm, \y2+-4mm) {};
\path[-](assignment_4) edge [] node [right]{} (loop_1_back_edge_angle1);
\draw let \p1=(loop_1_back_edge_angle1) in let \p2=(loop_1_back_edge_angle1) in node[coordinate](loop_1_back_edge_angle2) at (\x1+20mm, \y2+0mm) {};
\path[-](loop_1_back_edge_angle1) edge [] node [right]{} (loop_1_back_edge_angle2);
\draw let \p1=(loop_1_back_edge_angle2) in let \p2=(loop_1_join) in node[coordinate](loop_1_back_edge_angle3) at (\x1+0mm, \y2+0mm) {};
\path[-](loop_1_back_edge_angle2) edge [] node [right]{} (loop_1_back_edge_angle3);
\path[->](loop_1_back_edge_angle3) edge [] node [right]{} (loop_1_join);
\node[below=of loop_1_fork_angle_l.south, coordinate, yshift=3mm](stop){Stop};
\path[->](loop_1_fork_angle_l) edge [] node [left]{$x = {\displaystyle \sum_{k = 0}^{n} 5k}$} (stop);
	\end{tikzpicture}
\end{tabularx} 

\noindent Find suitable loop invariants and prove them locally consistent. Discuss, why these invariants have to be like that.

\end{assignment}


\clearpage
\begin{assignment}[L]{Two b, or not two b}
	Prove $Z$ using weakest preconditions.
	\begin{center}
		\begin{tikzpicture}
		\node[circle,draw](start){Start};
\node[below=of start.south, rectangle, draw, yshift=3mm](assignment_0){$x = 0$};
\path[->](start) edge [] node [right]{} (assignment_0);
\node[below=of assignment_0.south, rectangle, draw, yshift=3mm](read_1){$n = read()$};
\path[->](assignment_0) edge [] node [right]{} (read_1);
\node[below=of read_1.south, rectangle, draw, yshift=3mm](assignment_2){$i = 0$};
\path[->](read_1) edge [] node [right]{} (assignment_2);
\node[below=of assignment_2.south, rectangle, draw, yshift=3mm](assignment_3){$b = 0$};
\path[->](assignment_2) edge [] node [right]{} (assignment_3);
\node[below=of assignment_3.south, circle, draw, yshift=3mm](loop_4_join){};
\path[->](assignment_3) edge [] node [right]{} (loop_4_join);
\node[below=of loop_4_join.south, diamond, draw, yshift=3mm](loop_4_fork){i == n};
\path[->](loop_4_join) edge [] node [right]{} (loop_4_fork);
\draw let \p1=(loop_4_fork.west) in let \p2=(loop_4_fork.west) in node[coordinate](loop_4_fork_angle_l) at (\x1+-10mm, \y2+0mm) {};
\path[-](loop_4_fork) edge [] node [above]{$yes$} (loop_4_fork_angle_l);
\draw let \p1=(loop_4_fork.east) in let \p2=(loop_4_fork.east) in node[coordinate](loop_4_fork_angle_r) at (\x1+20mm, \y2+0mm) {};
\path[-](loop_4_fork) edge [] node [above]{$no$} (loop_4_fork_angle_r);
\node[below=of loop_4_fork_angle_l.south, rectangle, draw, yshift=3mm](write_5){$write(x)$};
\path[->](loop_4_fork_angle_l) edge [] node [left]{} (write_5);
\node[below=of loop_4_fork_angle_r.south, diamond, draw, yshift=3mm](branch_6_fork){b == 0};
\path[->](loop_4_fork_angle_r) edge [] node [right]{} (branch_6_fork);
\draw let \p1=(branch_6_fork.west) in let \p2=(branch_6_fork.west) in node[coordinate](branch_6_fork_angle_l) at (\x1+-10mm, \y2+0mm) {};
\path[-](branch_6_fork) edge [] node [above]{$no$} (branch_6_fork_angle_l);
\draw let \p1=(branch_6_fork.east) in let \p2=(branch_6_fork.east) in node[coordinate](branch_6_fork_angle_r) at (\x1+10mm, \y2+0mm) {};
\path[-](branch_6_fork) edge [] node [above]{$yes$} (branch_6_fork_angle_r);
\node[below=of branch_6_fork_angle_l.south, rectangle, draw, yshift=3mm](assignment_7){$x = x+3$};
\path[->](branch_6_fork_angle_l) edge [] node [right]{} (assignment_7);
\node[below=of assignment_7.south, rectangle, draw, yshift=3mm](assignment_8){$i = i+1$};
\path[->](assignment_7) edge [] node [right]{} (assignment_8);
\node[below=of branch_6_fork_angle_r.south, rectangle, draw, yshift=3mm](assignment_9){$x = x+2$};
\path[->](branch_6_fork_angle_r) edge [] node [right]{} (assignment_9);
\node[below=of assignment_8.south, coordinate, yshift=3mm](branch_6_join_anchor_l){};
\node[below=of assignment_9.south, coordinate, yshift=3mm](branch_6_join_anchor_r){};
\draw let \p1=(branch_6_join_anchor_l) in let \p2=(branch_6_join_anchor_r) in let \p3=(branch_6_fork) in node[circle,draw](branch_6_join) at (\x3,{min(\y1,\y2)}) {};\draw let \p1=(branch_6_join_anchor_l) in let \p2=(branch_6_join) in node[coordinate](branch_6_join_angle_l) at (\x1+0mm, \y2+0mm) {};
\path[-](assignment_8) edge [] node [right]{} (branch_6_join_angle_l);
\path[->](branch_6_join_angle_l) edge [] node [right]{} (branch_6_join);
\draw let \p1=(branch_6_join_anchor_r) in let \p2=(branch_6_join) in node[coordinate](branch_6_join_angle_r) at (\x1+0mm, \y2+0mm) {};
\path[-](assignment_9) edge [] node [right]{} (branch_6_join_angle_r);
\path[->](branch_6_join_angle_r) edge [] node [right]{} (branch_6_join);
\node[below=of branch_6_join.south, rectangle, draw, yshift=3mm](assignment_10){$b = 1-b$};
\path[->](branch_6_join) edge [] node [right]{} (assignment_10);
\draw let \p1=(assignment_10.south) in let \p2=(assignment_10.south) in node[coordinate](loop_4_back_edge_angle1) at (\x1+0mm, \y2+-4mm) {};
\path[-](assignment_10) edge [] node [right]{} (loop_4_back_edge_angle1);
\draw let \p1=(loop_4_back_edge_angle1) in let \p2=(loop_4_back_edge_angle1) in node[coordinate](loop_4_back_edge_angle2) at (\x1+40mm, \y2+0mm) {};
\path[-](loop_4_back_edge_angle1) edge [] node [right]{} (loop_4_back_edge_angle2);
\draw let \p1=(loop_4_back_edge_angle2) in let \p2=(loop_4_join) in node[coordinate](loop_4_back_edge_angle3) at (\x1+0mm, \y2+0mm) {};
\path[-](loop_4_back_edge_angle2) edge [] node [right]{} (loop_4_back_edge_angle3);
\path[->](loop_4_back_edge_angle3) edge [] node [right]{} (loop_4_join);
\node[below=of write_5.south, circle, draw,yshift=3mm](stop){Stop};
\path[->](write_5) edge [] node [left]{$\mbox{Z} :\equiv x = 5n$} (stop);

		\end{tikzpicture}
	\end{center}

\end{assignment}

\clearpage
\begin{assignment}[L]{Squared}
	Given is the following control flow graph:
	\begin{center}
		\begin{tikzpicture}\node[circle,draw](start){Start};
		\node[below=of start.south, rectangle, draw, yshift=3mm](assignment_0){$x = 0$};
		\path[->](start) edge [] node [right]{} (assignment_0);
		\node[below=of assignment_0.south, rectangle, draw, yshift=3mm](assignment_1){$i = 0$};
		\path[->](assignment_0) edge [] node [right]{} (assignment_1);
		\node[below=of assignment_1.south, rectangle, draw, yshift=3mm](read_2){$n = read()$};
		\path[->](assignment_1) edge [] node [right]{} (read_2);
		\node[below=of read_2.south, diamond, draw, yshift=3mm](branch_3_fork){n < 0};
		\path[->](read_2) edge [] node [right]{} (branch_3_fork);
		\draw let \p1=(branch_3_fork.west) in let \p2=(branch_3_fork.west) in node[coordinate](branch_3_fork_angle_l) at (\x1+-10mm, \y2+0mm) {};
		\path[-](branch_3_fork) edge [] node [above]{$no$} (branch_3_fork_angle_l);
		\draw let \p1=(branch_3_fork.east) in let \p2=(branch_3_fork.east) in node[coordinate](branch_3_fork_angle_r) at (\x1+10mm, \y2+0mm) {};
		\path[-](branch_3_fork) edge [] node [above]{$yes$} (branch_3_fork_angle_r);
		\node[below=of branch_3_fork_angle_l.south, rectangle, draw, yshift=3mm](assignment_4){$m = n$};
		\path[->](branch_3_fork_angle_l) edge [] node [right]{} (assignment_4);
		\node[below=of branch_3_fork_angle_r.south, rectangle, draw, yshift=3mm](assignment_5){$m = -n$};
		\path[->](branch_3_fork_angle_r) edge [] node [right]{} (assignment_5);
		\node[below=of assignment_4.south, coordinate, yshift=3mm](branch_3_join_anchor_l){};
		\node[below=of assignment_5.south, coordinate, yshift=3mm](branch_3_join_anchor_r){};
		\draw let \p1=(branch_3_join_anchor_l) in let \p2=(branch_3_join_anchor_r) in let \p3=(branch_3_fork) in node[circle,draw](branch_3_join) at (\x3,{min(\y1,\y2)}) {};\draw let \p1=(branch_3_join_anchor_l) in let \p2=(branch_3_join) in node[coordinate](branch_3_join_angle_l) at (\x1+0mm, \y2+0mm) {};
		\path[-](assignment_4) edge [] node [right]{} (branch_3_join_angle_l);
		\path[->](branch_3_join_angle_l) edge [] node [right]{} (branch_3_join);
		\draw let \p1=(branch_3_join_anchor_r) in let \p2=(branch_3_join) in node[coordinate](branch_3_join_angle_r) at (\x1+0mm, \y2+0mm) {};
		\path[-](assignment_5) edge [] node [right]{} (branch_3_join_angle_r);
		\path[->](branch_3_join_angle_r) edge [] node [right]{} (branch_3_join);
		\node[below=of branch_3_join.south, circle, draw, yshift=3mm](loop_6_join){};
		\path[->](branch_3_join) edge [] node [right]{} (loop_6_join);
		\node[below=of loop_6_join.south, diamond, draw, yshift=3mm](loop_6_fork){i == m};
		\path[->](loop_6_join) edge [] node [right]{} (loop_6_fork);
		\draw let \p1=(loop_6_fork.west) in let \p2=(loop_6_fork.west) in node[coordinate](loop_6_fork_angle_l) at (\x1+-10mm, \y2+0mm) {};
		\path[-](loop_6_fork) edge [] node [above]{$yes$} (loop_6_fork_angle_l);
		\draw let \p1=(loop_6_fork.east) in let \p2=(loop_6_fork.east) in node[coordinate](loop_6_fork_angle_r) at (\x1+10mm, \y2+0mm) {};
		\path[-](loop_6_fork) edge [] node [above]{$no$} (loop_6_fork_angle_r);
		\node[below=of loop_6_fork_angle_l.south, rectangle, draw, yshift=3mm](write_7){$write(x)$};
		\path[->](loop_6_fork_angle_l) edge [] node [left]{} (write_7);
		\node[below=of loop_6_fork_angle_r.south, rectangle, draw, yshift=3mm](assignment_8){$x = x+i$};
		\path[->](loop_6_fork_angle_r) edge [] node [right]{} (assignment_8);
		\node[below=of assignment_8.south, rectangle, draw, yshift=3mm](assignment_9){$i = i+1$};
		\path[->](assignment_8) edge [] node [right]{} (assignment_9);
		\node[below=of assignment_9.south, rectangle, draw, yshift=3mm](assignment_10){$x = x+i$};
		\path[->](assignment_9) edge [] node [right]{} (assignment_10);
		\draw let \p1=(assignment_10.south) in let \p2=(assignment_10.south) in node[coordinate](loop_6_back_edge_angle1) at (\x1+0mm, \y2+-10mm) {};
		\path[-](assignment_10) edge [] node [right]{} (loop_6_back_edge_angle1);
		\draw let \p1=(loop_6_back_edge_angle1) in let \p2=(loop_6_back_edge_angle1) in node[coordinate](loop_6_back_edge_angle2) at (\x1+20mm, \y2+0mm) {};
		\path[-](loop_6_back_edge_angle1) edge [] node [right]{} (loop_6_back_edge_angle2);
		\draw let \p1=(loop_6_back_edge_angle2) in let \p2=(loop_6_join) in node[coordinate](loop_6_back_edge_angle3) at (\x1+0mm, \y2+0mm) {};
		\path[-](loop_6_back_edge_angle2) edge [] node [right]{} (loop_6_back_edge_angle3);
		\path[->](loop_6_back_edge_angle3) edge [] node [right]{} (loop_6_join);
		\node[below=of write_7.south, circle, draw,yshift=3mm](stop){Stop};
		\path[->](write_7) edge [] node [left]{$Z \defequiv x = n^{2}$} (stop);
		\end{tikzpicture}
	\end{center}
	Prove that $Z$ holds.
	
\end{assignment}


\clearpage
\begin{assignment}[H, points=6]{Ready, Z, go!}
	Find a formula $Z$ to express the exact value $x$ the program computes. Then prove this $Z$ using weakest preconditions.
  \begin{center}
  \begin{tikzpicture}\node[circle,draw](start){Start};
\node[below=of start.south, rectangle, draw, yshift=3mm](read_0){$n = read()$};
\path[->](start) edge [] node [right]{$\mbox{}$} (read_0);
\node[below=of read_0.south, rectangle, draw, yshift=3mm](assignment_1){$i = 0$};
\path[->](read_0) edge [] node [right]{$\mbox{}$} (assignment_1);
\node[below=of assignment_1.south, rectangle, draw, yshift=3mm](assignment_2){$x = 1$};
\path[->](assignment_1) edge [] node [right]{$\mbox{}$} (assignment_2);
\node[below=of assignment_2.south, rectangle, draw, yshift=3mm](assignment_3){$y = 0$};
\path[->](assignment_2) edge [] node [right]{$\mbox{}$} (assignment_3);
\node[below=of assignment_3.south, circle, draw, yshift=3mm](loop_4_join){};
\path[->](assignment_3) edge [] node [right]{$\mbox{}$} (loop_4_join);
\node[below=of loop_4_join.south, diamond, draw, yshift=3mm](loop_4_fork){i == n};
\path[->](loop_4_join) edge [] node [right]{$\mbox{}$} (loop_4_fork);
\draw let \p1=(loop_4_fork.west) in let \p2=(loop_4_fork.west) in node[coordinate](loop_4_fork_angle_l) at (\x1+-10mm, \y2+0mm) {};
\path[-](loop_4_fork) edge [] node [above]{$yes$} (loop_4_fork_angle_l);
\draw let \p1=(loop_4_fork.east) in let \p2=(loop_4_fork.east) in node[coordinate](loop_4_fork_angle_r) at (\x1+10mm, \y2+0mm) {};
\path[-](loop_4_fork) edge [] node [above]{$no$} (loop_4_fork_angle_r);
\node[below=of loop_4_fork_angle_l.south, rectangle, draw, yshift=3mm](write_5){$write(x)$};
\path[->](loop_4_fork_angle_l) edge [] node [left]{$\mbox{}$} (write_5);
\node[below=of loop_4_fork_angle_r.south, rectangle, draw, yshift=3mm](assignment_6){$y = y+3$};
\path[->](loop_4_fork_angle_r) edge [] node [right]{$\mbox{}$} (assignment_6);
\node[below=of assignment_6.south, rectangle, draw, yshift=3mm](assignment_7){$x = y*x$};
\path[->](assignment_6) edge [] node [right]{$\mbox{}$} (assignment_7);
\node[below=of assignment_7.south, rectangle, draw, yshift=3mm](assignment_8){$i = i+1$};
\path[->](assignment_7) edge [] node [right]{$\mbox{}$} (assignment_8);
\draw let \p1=(assignment_8.south) in let \p2=(assignment_8.south) in node[coordinate](loop_4_back_edge_angle1) at (\x1+0mm, \y2+-10mm) {};
\path[-](assignment_8) edge [] node [right]{$\mbox{}$} (loop_4_back_edge_angle1);
\draw let \p1=(loop_4_back_edge_angle1) in let \p2=(loop_4_back_edge_angle1) in node[coordinate](loop_4_back_edge_angle2) at (\x1+20mm, \y2+0mm) {};
\path[-](loop_4_back_edge_angle1) edge [] node [right]{} (loop_4_back_edge_angle2);
\draw let \p1=(loop_4_back_edge_angle2) in let \p2=(loop_4_join) in node[coordinate](loop_4_back_edge_angle3) at (\x1+0mm, \y2+0mm) {};
\path[-](loop_4_back_edge_angle2) edge [] node [right]{} (loop_4_back_edge_angle3);
\path[->](loop_4_back_edge_angle3) edge [] node [right]{} (loop_4_join);
\node[below=of write_5.south, circle, draw,yshift=3mm](stop){Stop};
\path[->](write_5) edge [] node [left]{$\mbox{Z}$} (stop);
  \end{tikzpicture}
  \end{center}


\end{assignment}

\clearpage
\begin{assignment}[H,points=8]{Loloopop}
	Prove $Z$ using weakest preconditions:
	\begin{center}
		\begin{tikzpicture}
			\node[circle,draw](start){Start};
\node[below=of start.south, rectangle, draw, yshift=3mm](assignment_0){$x = 1$};
\path[->](start) edge [] node [right]{} (assignment_0);
\node[below=of assignment_0.south, rectangle, draw, yshift=3mm](assignment_1){$i = 0$};
\path[->](assignment_0) edge [] node [right]{} (assignment_1);
\node[below=of assignment_1.south, rectangle, draw, yshift=3mm](read_2){$n = read()$};
\path[->](assignment_1) edge [] node [right]{} (read_2);
\node[below=of read_2.south, circle, draw, yshift=3mm](loop_3_join){};
\path[->](read_2) edge [] node [right]{} (loop_3_join);
\node[below=of loop_3_join.south, diamond, draw, yshift=3mm](loop_3_fork){i == n};
\path[->](loop_3_join) edge [] node [right]{} (loop_3_fork);
\draw let \p1=(loop_3_fork.west) in let \p2=(loop_3_fork.west) in node[coordinate](loop_3_fork_angle_l) at (\x1+-10mm, \y2+0mm) {};
\path[-](loop_3_fork) edge [] node [above]{$yes$} (loop_3_fork_angle_l);
\draw let \p1=(loop_3_fork.east) in let \p2=(loop_3_fork.east) in node[coordinate](loop_3_fork_angle_r) at (\x1+10mm, \y2+0mm) {};
\path[-](loop_3_fork) edge [] node [above]{$no$} (loop_3_fork_angle_r);
\node[below=of loop_3_fork_angle_l.south, rectangle, draw, yshift=3mm](write_4){$write(x)$};
\path[->](loop_3_fork_angle_l) edge [] node [left]{} (write_4);
\node[below=of loop_3_fork_angle_r.south, rectangle, draw, yshift=3mm](assignment_5){$i = i+1$};
\path[->](loop_3_fork_angle_r) edge [] node [right]{} (assignment_5);
\node[below=of assignment_5.south, rectangle, draw, yshift=3mm](assignment_6){$j = i$};
\path[->](assignment_5) edge [] node [right]{} (assignment_6);
\node[below=of assignment_6.south, circle, draw, yshift=3mm](loop_7_join){};
\path[->](assignment_6) edge [] node [right]{} (loop_7_join);
\node[below=of loop_7_join.south, diamond, draw, yshift=3mm](loop_7_fork){j > 0};
\path[->](loop_7_join) edge [] node [right]{} (loop_7_fork);
\draw let \p1=(loop_7_fork.west) in let \p2=(loop_7_fork.west) in node[coordinate](loop_7_fork_angle_l) at (\x1+-10mm, \y2+0mm) {};
\path[-](loop_7_fork) edge [] node [above]{$no$} (loop_7_fork_angle_l);
\draw let \p1=(loop_7_fork.east) in let \p2=(loop_7_fork.east) in node[coordinate](loop_7_fork_angle_r) at (\x1+10mm, \y2+0mm) {};
\path[-](loop_7_fork) edge [] node [above]{$yes$} (loop_7_fork_angle_r);
\node[below=of loop_7_fork_angle_r.south, rectangle, draw, yshift=3mm](assignment_8){$x = x*i$};
\path[->](loop_7_fork_angle_r) edge [] node [right]{} (assignment_8);
\node[below=of assignment_8.south, rectangle, draw, yshift=3mm](assignment_9){$j = j-1$};
\path[->](assignment_8) edge [] node [right]{} (assignment_9);
\draw let \p1=(assignment_9.south) in let \p2=(assignment_9.south) in node[coordinate](loop_7_back_edge_angle1) at (\x1+0mm, \y2+-6mm) {};
\path[-](assignment_9) edge [] node [right]{} (loop_7_back_edge_angle1);
\draw let \p1=(loop_7_back_edge_angle1) in let \p2=(loop_7_back_edge_angle1) in node[coordinate](loop_7_back_edge_angle2) at (\x1+14mm, \y2+0mm) {};
\path[-](loop_7_back_edge_angle1) edge [] node [right]{} (loop_7_back_edge_angle2);
\draw let \p1=(loop_7_back_edge_angle2) in let \p2=(loop_7_join) in node[coordinate](loop_7_back_edge_angle3) at (\x1+0mm, \y2+0mm) {};
\path[-](loop_7_back_edge_angle2) edge [] node [right]{} (loop_7_back_edge_angle3);
\path[->](loop_7_back_edge_angle3) edge [] node [right]{} (loop_7_join);
\draw let \p1=(loop_7_fork_angle_l.south) in let \p2=(loop_7_fork_angle_l.south) in node[coordinate](loop_3_back_edge_angle1) at (\x1+0mm, \y2+-40mm) {};
\path[-](loop_7_fork_angle_l) edge [] node [left]{} (loop_3_back_edge_angle1);
\draw let \p1=(loop_3_back_edge_angle1) in let \p2=(loop_3_back_edge_angle1) in node[coordinate](loop_3_back_edge_angle2) at (\x1+62mm, \y2+0mm) {};
\path[-](loop_3_back_edge_angle1) edge [] node [right]{} (loop_3_back_edge_angle2);
\draw let \p1=(loop_3_back_edge_angle2) in let \p2=(loop_3_join) in node[coordinate](loop_3_back_edge_angle3) at (\x1+0mm, \y2+0mm) {};
\path[-](loop_3_back_edge_angle2) edge [] node [right]{} (loop_3_back_edge_angle3);
\path[->](loop_3_back_edge_angle3) edge [] node [right]{} (loop_3_join);
\node[below=of write_4.south, circle, draw,yshift=3mm](stop){Stop};
\path[->](write_4) edge [] node [left]{$\mbox{Z} :\equiv x = {\displaystyle \prod_{k = 1}^{n} (k)^{k}}$} (stop);

		\end{tikzpicture}
	\end{center}

	\noindent \hint{If you have to find invariants for nested loops, it is usually easiest to work from outermost loop to innermost loop.}

\end{assignment}

\clearpage
\begin{assignment}[H, points=3]{Somethng s wrong wth ths program...}
	Prove $Z$ using weakest preconditions.
	\begin{center}
		\begin{tikzpicture}
			\node[circle,draw](start){Start};
\node[below=of start.south, rectangle, draw, yshift=3mm](assignment_0){$x = 0$};
\path[->](start) edge [] node [right]{} (assignment_0);
\node[below=of assignment_0.south, rectangle, draw, yshift=3mm](assignment_1){$y = 1$};
\path[->](assignment_0) edge [] node [right]{} (assignment_1);
\node[below=of assignment_1.south, circle, draw, yshift=3mm](loop_2_join){};
\path[->](assignment_1) edge [] node [right]{} (loop_2_join);
\node[below=of loop_2_join.south, diamond, draw, yshift=3mm](loop_2_fork){y == 1024};
\path[->](loop_2_join) edge [] node [right]{} (loop_2_fork);
\draw let \p1=(loop_2_fork.west) in let \p2=(loop_2_fork.west) in node[coordinate](loop_2_fork_angle_l) at (\x1+-10mm, \y2+0mm) {};
\path[-](loop_2_fork) edge [] node [above]{$yes$} (loop_2_fork_angle_l);
\draw let \p1=(loop_2_fork.east) in let \p2=(loop_2_fork.east) in node[coordinate](loop_2_fork_angle_r) at (\x1+10mm, \y2+0mm) {};
\path[-](loop_2_fork) edge [] node [above]{$no$} (loop_2_fork_angle_r);
\node[below=of loop_2_fork_angle_l.south, rectangle, draw, yshift=3mm](write_3){$write(x)$};
\path[->](loop_2_fork_angle_l) edge [] node [left]{} (write_3);
\node[below=of loop_2_fork_angle_r.south, rectangle, draw, yshift=3mm](assignment_4){$y = y*2$};
\path[->](loop_2_fork_angle_r) edge [] node [right]{} (assignment_4);
\node[below=of assignment_4.south, rectangle, draw, yshift=3mm](assignment_5){$x = x+10$};
\path[->](assignment_4) edge [] node [right]{} (assignment_5);
\draw let \p1=(assignment_5.south) in let \p2=(assignment_5.south) in node[coordinate](loop_2_back_edge_angle1) at (\x1+0mm, \y2+-10mm) {};
\path[-](assignment_5) edge [] node [right]{} (loop_2_back_edge_angle1);
\draw let \p1=(loop_2_back_edge_angle1) in let \p2=(loop_2_back_edge_angle1) in node[coordinate](loop_2_back_edge_angle2) at (\x1+20mm, \y2+0mm) {};
\path[-](loop_2_back_edge_angle1) edge [] node [right]{} (loop_2_back_edge_angle2);
\draw let \p1=(loop_2_back_edge_angle2) in let \p2=(loop_2_join) in node[coordinate](loop_2_back_edge_angle3) at (\x1+0mm, \y2+0mm) {};
\path[-](loop_2_back_edge_angle2) edge [] node [right]{} (loop_2_back_edge_angle3);
\path[->](loop_2_back_edge_angle3) edge [] node [right]{} (loop_2_join);
\node[below=of write_3.south, circle, draw,yshift=3mm](stop){Stop};
\path[->](write_3) edge [] node [left]{$\mbox{Z} :\equiv x = 100$} (stop);

		\end{tikzpicture}
	\end{center}

\end{assignment}

\begin{assignment}[H,points=3]{A Neverending Story}
Prove that the following program cannot terminate using weakest preconditions.
\begin{center}
\begin{tikzpicture}
	\node[draw,circle](start) {Start};
	\node[draw,rectangle,below=of start](read) {$n = read()$};
	\node[draw,diamond,below=of read](branch) {$n > 0$};
	\node[coordinate,left=of branch,xshift=-3mm](branch_angle_l) {};
	\node[draw,circle,below=of branch_angle_l](join_l) {};
	\node[draw,diamond,below=of join_l](cond_l) {$n <= 0$};
	\node[coordinate,left=of cond_l](cond_l_angle) {};
	\node[draw,rectangle,below=of cond_l_angle](assign_l) {$n = n - 1$};
	\node[coordinate,below=of assign_l,yshift=3mm](back_l_angle_0) {};
	\node[coordinate,left=of back_l_angle_0,xshift=-5mm](back_l_angle_1) {};
	\draw let \p1=(back_l_angle_1) in let \p2=(join_l) in node[coordinate](back_l_angle_2) at (\x1,\y2) {};

	\path[->](start) edge [] node [] {} (read);
	\path[->](read) edge [] node [] {} (branch);

	\path[-](branch) edge [] node[above]{no} (branch_angle_l);
	\path[->](branch_angle_l) edge [] node[] {} (join_l);
	\path[->](join_l) edge [] node[] {} (cond_l);
	\path[-](cond_l) edge [] node[above] {yes} (cond_l_angle);
	\path[->](cond_l_angle) edge [] node[] {} (assign_l);
	\path[-](assign_l) edge [] node[] {} (back_l_angle_0);
	\path[-](back_l_angle_0) edge [] node[] {} (back_l_angle_1);
	\path[-](back_l_angle_1) edge [] node[] {} (back_l_angle_2);
	\path[->](back_l_angle_2) edge [] node[] {} (join_l);

	\node[coordinate,right=of branch,xshift=3mm](branch_angle_r) {};
	\node[draw,circle,below=of branch_angle_r](join_r) {};
	\node[draw,diamond,below=of join_r](cond_r) {$n <= 0$};
	\node[coordinate,right=of cond_r](cond_r_angle) {};
	\node[draw,rectangle,below=of cond_r_angle](assign_r) {$n = n + 1$};
	\node[coordinate,below=of assign_r,yshift=3mm](back_r_angle_0) {};
	\node[coordinate,right=of back_r_angle_0,xshift=5mm](back_r_angle_1) {};
	\draw let \p1=(back_r_angle_1) in let \p2=(join_r) in node[coordinate](back_r_angle_2) at (\x1,\y2) {};

	\path[-](branch) edge [] node[above]{yes} (branch_angle_r);
	\path[->](branch_angle_r) edge [] node[] {} (join_r);
	\path[->](join_r) edge [] node[] {} (cond_r);
	\path[-](cond_r) edge [] node[above] {no} (cond_r_angle);
	\path[->](cond_r_angle) edge [] node[] {} (assign_r);
	\path[-](assign_r) edge [] node[] {} (back_r_angle_0);
	\path[-](back_r_angle_0) edge [] node[] {} (back_r_angle_1);
	\path[-](back_r_angle_1) edge [] node[] {} (back_r_angle_2);
	\path[->](back_r_angle_2) edge [] node[] {} (join_r);

	\draw let \p1=(branch) in let \p2=(cond_r) in node[draw,circle](join) at (\x1,\y2) {};
	\node[draw,circle,below=of join](stop) {Stop};

	\path[->](cond_l) edge [] node[above] {no} (join);
	\path[->](cond_r) edge [] node[above] {yes} (join);
	\path[->](join) edge [] node[] {} (stop);

\end{tikzpicture}
\end{center}


\end{assignment}

\end{document}





































